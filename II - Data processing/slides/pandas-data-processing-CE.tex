\documentclass[aspectratio=169]{beamer}

\usetheme{metropolis}           % Use metropolis theme
\usepackage[utf8]{inputenc}
\usepackage{graphicx}
\usepackage{eso-pic}
\usepackage{graphics}
\usepackage{tikz}
\usepackage[export]{adjustbox}
\usepackage{multicol}
\usepackage{listings}
\usepackage{helvet}
\usepackage{booktabs}
\usepackage{threeparttable}
\usepackage{marvosym}
\usepackage{hyperref}
\usepackage{xcolor}
\usepackage{soul}	% For strike-through
\usepackage{tcolorbox} % For color box

\newcounter{exercises}

\title{Data Processing in Python\\Using Pandas}
\date{}
\author{DIME Analytics \newline Presented by Luis Eduardo San Martin} % Name of author(s) of session here
\institute{Development Impact Evaluation (DIME) \newline The World Bank }
\setbeamercolor{background canvas}{bg=white}	% Sets background color

% The below command places the World Bank logo and DIME logo to the right corner
\titlegraphic{%
	\begin{picture}(0,0)
	\put(330,-180){\makebox(0,0)[rt]{\includegraphics[width=3cm]{img/WB_logo}}}
	\end{picture}%
	\begin{picture}(0,0)
	\put(390,-180){\makebox(0,0)[rt]{\includegraphics[width=1.5cm]{img/i2i}}}
	\end{picture}%
}

%%% Section page with picture of Light bulb
\makeatletter
\defbeamertemplate*{section page}{mytheme}[1][]{
	\centering
	\begin{minipage}{22em}
		\raggedright
		\usebeamercolor[fg]{section title}
		\usebeamerfont{section title}
		\par
		\ifx\insertsubsectionhead\@empty\else%
		\usebeamercolor[fg]{subsection title}%
		\usebeamerfont{subsection title}%
		\fi
		\ifstrempty{#1}{}{%
			\includegraphics[width=100mm, height=60mm]{#1}%
		}
		\\
		\insertsectionhead\\[-1ex]
		\insertsubsectionhead
		\usebeamertemplate*{progress bar in section page}
		
	\end{minipage}
	\par
	\vspace{\baselineskip}
}
\makeatother

%%% Define a command to include picture in section, 
%%% make section, and revert to old template
\newcommand{\sectionpic}[2]{
	\setbeamertemplate{section page}[mytheme][#2]
	\section{#1}
	\setbeamertemplate{section page}[mytheme]
}

%%% The command below allows for the text that contains Stata code
\lstset{ %
	backgroundcolor=\color{white},
	basicstyle=\tiny,
	breakatwhitespace=false,
	breaklines=true,
	captionpos=b,
	commentstyle=\color{green},
	escapeinside={\%*}{*)},
	extendedchars=true,
	frame=single,
	numbers=left,
	numbersep=5pt,
	numberstyle=\tiny\color{gray},
	rulecolor=\color{black},
	showspaces=false,
	showstringspaces=false,
	showtabs=false,
	stringstyle=\color{mauve},
	tabsize=2,
	title=\lstname,
	morekeywords={not,\},\{,preconditions,effects },
	deletekeywords={time}
}

%% The below command creates the ligh bulb logos in the top right corner of the 
\begin{document}
	
	
	%%%%%%%%%%%%%%%%%%%%%%%%%%%%%%%%%%%%%%%%%%% Title slide
	{
		\usebackgroundtemplate{\includegraphics[height=55mm,right]{img/top_right_corner.pdf}} 
		\maketitle
	}

\begin{frame}

	\frametitle{Overview} % Table of contents slide, comment this block out to remove it
	\tableofcontents % Throughout your presentation, if you choose to use \section{} and \subsection{} commands, these will automatically be printed on this slide as an overview of your presentation

\end{frame}

%%%%%%%%%%%%%%%%%%%%%%%%%%%%%%%%%%%%%%%%%%% Section title slide
\sectionpic{Introduction}{img/section_slide}

%%%%%%%%%%%%%%%%%%%%%%%%%%%%%%%%%%%%%%%%%%% Slide
\begin{frame}{Introduction}

	\begin{itemize}
		\item This session will introduce you to Pandas, the most popular data frames processing library in Python
		\item We'll review:
		\begin{itemize}
			\item Pandas dataframes
			\item Data processing operations
		\end{itemize}
		\item We'll take advantage of your Stata knowledge to establish comparisons between Stata's and Pandas' commands
	\end{itemize}

\end{frame}

%%%%%%%%%%%%%%%%%%%%%%%%%%%%%%%%%%%%%%%%%%% Slide
\begin{frame}{Introduction - Why Pandas if I already know Stata?}

	\begin{itemize}
		\item \textbf{Big data:} though Pandas is not suitable for big data,  the most popular Python big data tools expect you to know it
		\item \textbf{ML:} If you ever want to implement a ML pipeline using Python, you'll need Pandas for your data transformation
		\item \textbf{Cloud platforms:} Cloud platforms assume Python much more often than Stata, and Pandas is the library assumed for data processing
		\item \textbf{General Python data work:} Being the go-to library for data processing, most Python data work libraries assume Pandas knowledge -- for example: many of them use Pandas dataframes or series objects as inputs
	\end{itemize}

\end{frame}

%%%%%%%%%%%%%%%%%%%%%%%%%%%%%%%%%%%%%%%%%%% Section title slide
\sectionpic{Getting Started}{img/section_slide}

%%%%%%%%%%%%%%%%%%%%%%%%%%%%%%%%%%%%%%%%%%% Slide
\begin{frame}{Getting started}

	\begin{itemize}
		\item We'll use \href{http://colab.research.google.com}{Google Colab} today
		\item It is similar to a Google Doc but for coding, and runs Python by default
	\end{itemize}

\end{frame}

%%%%%%%%%%%%%%%%%%%%%%%%%%%%%%%%%%%%%%%%%%% Slide
\begin{frame}{Getting started}

	\begin{itemize}
		\item Go to https://colab.research.google.com
		\item Click on \texttt{NEW NOTEBOOK} if you're already logged in, or go to \texttt{File > New notebook} if you're not
	\end{itemize}

	\begin{multicols}{2}

		\begin{figure}
			\centering
			\includegraphics[width=0.9\linewidth]{img/new_nb_logged_in.png}
			\caption{Do this if you're already logged in}
		\end{figure}
		\begin{figure}
			\centering
			\includegraphics[width=0.6\linewidth]{img/new_nb_not_logged_in.png}
			\caption{Do this if you're not -- you'll be prompted to log in}
		\end{figure}

	\end{multicols}

\end{frame}

%%%%%%%%%%%%%%%%%%%%%%%%%%%%%%%%%%%%%%%%%%% Section title slide
\sectionpic{The Pandas library}{img/section_slide}

%%%%%%%%%%%%%%%%%%%%%%%%%%%%%%%%%%%%%%%%%%% Slide
\begin{frame}{Pandas}

	\begin{itemize}
		\item Pandas is used for data processing and data analysis
		\item Use the following command to import Pandas to your notebook:
	\end{itemize}

	\hspace{7mm} \texttt{\textcolor{purple}{import} pandas as \textcolor{purple}{pd}}

\end{frame}

%%%%%%%%%%%%%%%%%%%%%%%%%%%%%%%%%%%%%%%%%%% Subsection
\subsection{Pandas dataframes}

%%%%%%%%%%%%%%%%%%%%%%%%%%%%%%%%%%%%%%%%%%% Slide
\begin{frame}{Pandas}

	\textbf{Dataframes}

	\begin{multicols}{2}
	
		\begin{itemize}
			\item A dataframe is a two-dimensional data structure
			\item They are very often used to work with data structures where every row represents a unit of observation and every column represents an observation's attribute
			\item The Stata equivalent to dataframes are datasets
		\end{itemize}
		\begin{figure}
			\centering
			\includegraphics[width=0.8\linewidth]{img/dataframe.png}
		\end{figure}

	\end{multicols}

\end{frame}

%%%%%%%%%%%%%%%%%%%%%%%%%%%%%%%%%%%%%%%%%%% Slide
\begin{frame}{Pandas}

	\textbf{Creating a dataframe}

	There are several ways to create a dataframe from scratch. One of the easiest is:

	\begin{enumerate}
		\item Define a list of strings with the column names
		\begin{figure}
			\includegraphics[width=0.8\linewidth]{img/column_names.png}
		\end{figure}
		\item Define a list for \textbf{each observation}
		\begin{figure}
			\includegraphics[width=0.8\linewidth]{img/observation_lists.png}
		\end{figure}
		\item Wrap all of the observation lists in another list
		\begin{figure}
			\includegraphics[width=0.8\linewidth]{img/data_list.png}
		\end{figure}

	\end{enumerate}

\end{frame}

%%%%%%%%%%%%%%%%%%%%%%%%%%%%%%%%%%%%%%%%%%% Slide
\begin{frame}{Pandas}

	\textbf{Creating a dataframe}

	\begin{enumerate}
		\setcounter{enumi}{3}
		\item Use the lists \texttt{data} and \texttt{column\_names} as inputs in the \texttt{DataFrame()} command
		\begin{figure}
			\includegraphics[width=0.8\linewidth]{img/dataframe_definition.png}
		\end{figure}
		\item Now your dataframe is defined in the variable \texttt{df}. You can use that name to refer to it or print it, as in:
		\begin{figure}
			\includegraphics[width=0.8\linewidth]{img/df.png}
		\end{figure}
	\end{enumerate}

\end{frame}

%%%%%%%%%%%%%%%%%%%%%%%%%%%%%%%%%%%%%%%%%%% Slide
\begin{frame}{Pandas}

	\textbf{Creating a dataframe}

	Another way to create a dataframe from zero is to define an empty dataframe and then create its columns individually.
	\begin{figure}
		\includegraphics[width=0.8\linewidth]{img/df_definition2.png}
	\end{figure}

\end{frame}



%%%%%%%%%%%%%%%%%%%%%%%%%%%%%%%%%%%%%%%%%%% Section title slide
\sectionpic{Data processing}{img/section_slide}

%%%%%%%%%%%%%%%%%%%%%%%%%%%%%%%%%%%%%%%%%%% Subsection
\subsection{Reading data}

%%%%%%%%%%%%%%%%%%%%%%%%%%%%%%%%%%%%%%%%%%% Slide
\begin{frame}{Data processing}

	\textbf{Reading data}

	\begin{itemize}
		\item In our work, we don't usually need to define a dataframe from scratch
		\item More often, we load pre-existing data files
		\item To load a \texttt{.csv} file into a dataframe, we use the command \texttt{read\_csv()}
	\end{itemize}

\end{frame}

%%%%%%%%%%%%%%%%%%%%%%%%%%%%%%%%%%%%%%%%%%% Slide
\begin{frame}{Data processing}

	\textbf{Reading data}

	\texttt{read\_csv()}

	\begin{figure}
		\centering
		\includegraphics[width=0.9\linewidth]{img/read_csv.png}
	\end{figure}

\end{frame}

%%%%%%%%%%%%%%%%%%%%%%%%%%%%%%%%%%%%%%%%%%% Slide
\begin{frame}{Data processing}

	\textbf{Reading data}

	\texttt{read\_csv()}

	\begin{figure}
		\centering
		\includegraphics[width=0.9\linewidth]{img/read_csv.png}
	\end{figure}
	\begin{itemize}
		\item \texttt{data\_location} is a string with the location of our file
		\item It can be a URL or a path in your local disk -- though a path in your local disk won't work directly with Colab
		\item \texttt{read\_csv()} is the Pandas function to read \texttt{.csv} files into dataframes. It takes the file location string as input
	\end{itemize}

\end{frame}

%%%%%%%%%%%%%%%%%%%%%%%%%%%%%%%%%%%%%%%%%%% Slide
\begin{frame}{Data processing}

	\textbf{Reading data}

	\begin{figure}
		\centering
		\includegraphics[width=0.4\linewidth]{img/crops_df.png}
	\end{figure}

\end{frame}

%%%%%%%%%%%%%%%%%%%%%%%%%%%%%%%%%%%%%%%%%%% Subsection
\subsection{Exploring a dataframe}

%%%%%%%%%%%%%%%%%%%%%%%%%%%%%%%%%%%%%%%%%%% Slide
\begin{frame}{Data processing}

	\textbf{Exploring a dataframe}

	Running the dataframe name as a command will print it. If the dataframe has too many rows or columns, Python will print only the first and last rows and columns.

	\begin{figure}
		\includegraphics[width=0.4\linewidth]{img/crops_df.png}
	\end{figure}

\end{frame}

%%%%%%%%%%%%%%%%%%%%%%%%%%%%%%%%%%%%%%%%%%% Slide
\begin{frame}{Data processing}

	\textbf{Exploring a dataframe}

	We can also use the \texttt{.head()} and \texttt{.tail()} attributes to return the first and last observations of a dataframe

	\begin{multicols}{2}

		\begin{figure}
			\includegraphics[width=\linewidth]{img/head.png}
		\end{figure}
		\begin{figure}
			\includegraphics[width=\linewidth]{img/tail.png}
		\end{figure}

	\end{multicols}

\end{frame}

%%%%%%%%%%%%%%%%%%%%%%%%%%%%%%%%%%%%%%%%%%% Slide
\begin{frame}{Data processing}

	\textbf{Exploring a dataframe}

	To see how many rows and columns a dataframe has, we use the \texttt{.shape} attribute:
	
	\begin{figure}
		\includegraphics[width=0.7\linewidth]{img/crops_shape.png}
	\end{figure}

	The result is a tuple (an immutable list) whose elements are the number of rows and columns

\end{frame}

%%%%%%%%%%%%%%%%%%%%%%%%%%%%%%%%%%%%%%%%%%% Slide
\begin{frame}{Data processing}

	\textbf{Exploring a dataframe}

	We can also get the number of rows with the \texttt{len()} function:
	
	\begin{figure}
		\includegraphics[width=0.7\linewidth]{img/crops_len.png}
	\end{figure}
	
	The result of \texttt{len()} is an integer.

\end{frame}

%%%%%%%%%%%%%%%%%%%%%%%%%%%%%%%%%%%%%%%%%%% Slide
\begin{frame}{Data processing}

	\textbf{Exploring a dataframe}

	To check the column names, we use the \texttt{.columns} attribute:
	
	\begin{figure}
		\includegraphics[width=0.9\linewidth]{img/crops_columns.png}
	\end{figure}

\end{frame}

%%%%%%%%%%%%%%%%%%%%%%%%%%%%%%%%%%%%%%%%%%% Subsection
\subsection{Dataframe indexing}

%%%%%%%%%%%%%%%%%%%%%%%%%%%%%%%%%%%%%%%%%%% Slide
\begin{frame}{Data processing}

	\textbf{Dataframe indexing}

	\begin{multicols}{2}

		\begin{itemize}
			\item Every row and column of a dataframe has a \textbf{label}
			\item Row labels are represented by the index
			\item Column labels are represented by the column names
		\end{itemize}
		\begin{figure}
			\centering
			\includegraphics[width=0.71\linewidth]{img/dataframe_indices.png}
		\end{figure}

	\end{multicols}

\end{frame}

%%%%%%%%%%%%%%%%%%%%%%%%%%%%%%%%%%%%%%%%%%% Slide
\begin{frame}{Data processing}

	\textbf{Column indexing}

	We can subset a single column of a dataframe using two methods:

	\begin{itemize}
		\item \texttt{df.column\_name}
		\item \texttt{df['column\_name']}
	\end{itemize}

\end{frame}

%%%%%%%%%%%%%%%%%%%%%%%%%%%%%%%%%%%%%%%%%%% Slide
\begin{frame}{Data processing}

	\textbf{Column indexing}

	\begin{multicols}{2}

		\begin{figure}
			\centering
			\includegraphics[width=0.95\linewidth]{img/col_subset1.png}
		\end{figure}
		\begin{figure}
			\centering
			\includegraphics[width=0.99\linewidth]{img/col_subset2.png}
		\end{figure}

	\end{multicols}

\end{frame}

%%%%%%%%%%%%%%%%%%%%%%%%%%%%%%%%%%%%%%%%%%% Slide
\begin{frame}{Data processing}

	\textbf{Column indexing}

	The difference between the two is that the first method doesn't allow column name references, while the second does
	\begin{figure}
		\centering
		\includegraphics[width=0.47\linewidth]{img/col_name_reference.png}
	\end{figure}

\end{frame}

%%%%%%%%%%%%%%%%%%%%%%%%%%%%%%%%%%%%%%%%%%% Slide
\begin{frame}{Data processing}

	\textbf{Multi-column indexing}

	\begin{itemize}
		\item We can use a syntax similar to the second method to index more than one column at the same time
		\item Instead of including a string with one column name inside the brackets, we include a list of strings with the column names to index
	\end{itemize}

	\hspace{7mm} \texttt{df[['col\_name1', 'col\_name2', 'col\_name3', ...]]}

	\begin{itemize}
		\item Note that inside the outer brackets we have a list of strings
	\end{itemize}

\end{frame}

%%%%%%%%%%%%%%%%%%%%%%%%%%%%%%%%%%%%%%%%%%% Slide
\begin{frame}{Data processing}

	\textbf{Multi-column indexing}

	\begin{multicols}{2}

		\begin{figure}
			\centering
			\includegraphics[width=0.8\linewidth]{img/multicol_subsetting1.png}
		\end{figure}
		\begin{figure}
			\centering
			\includegraphics[width=0.75\linewidth]{img/multicol_subsetting2.png}
		\end{figure}

	\end{multicols}

\end{frame}

%%%%%%%%%%%%%%%%%%%%%%%%%%%%%%%%%%%%%%%%%%% Slide
\begin{frame}{Data processing}

	\textbf{Multi-column indexing}

	A one-column index operation returns a Pandas series. A multicolumn index returns another dataframe.

	\begin{multicols}{2}

		\begin{figure}
			\centering
			\includegraphics[width=0.7\linewidth]{img/price.png}
		\end{figure}
		\begin{figure}
			\centering
			\includegraphics[width=0.7\linewidth]{img/price_quantity.png}
		\end{figure}

	\end{multicols}

\end{frame}

%%%%%%%%%%%%%%%%%%%%%%%%%%%%%%%%%%%%%%%%%%% Slide
\begin{frame}{Data processing}

	\textbf{Row indexing}

	There are basically two methods to index rows in Pandas. The simplest is \texttt{.iloc[]}, which is used to index the i-th row or rows:

	\begin{itemize}
		\item Indexing a single row: \texttt{df.iloc[i]}
		\item Indexing a range of continuous rows from i until (j-1): \texttt{df.iloc[i:j]}
		\item Indexing the i-th and j-th non-continuous rows: \texttt{df.iloc[[i, j]]}
	\end{itemize}

	\textbf{Very important:} In Python, every numeric index starts at zero, not at one

\end{frame}

%%%%%%%%%%%%%%%%%%%%%%%%%%%%%%%%%%%%%%%%%%% Slide
\begin{frame}{Data processing}

	\textbf{Row indexing}

	\begin{itemize}
		\item Indexing a single row: \texttt{df.iloc[i]}
	\end{itemize}

	\begin{figure}
		\centering
		\includegraphics[width=0.6\linewidth]{img/single_row.png}
	\end{figure}

	Note that the outcome of indexing a single row is a Pandas series, not a dataframe

\end{frame}

%%%%%%%%%%%%%%%%%%%%%%%%%%%%%%%%%%%%%%%%%%% Slide
\begin{frame}{Data processing}

	\textbf{Row indexing}

	\begin{itemize}
		\item Indexing a range of continuous rows from i until (j-1): \texttt{df.iloc[i:j]}
	\end{itemize}

	\begin{figure}
		\centering
		\includegraphics[width=0.5\linewidth]{img/consecutive_rows.png}
	\end{figure}

\end{frame}

%%%%%%%%%%%%%%%%%%%%%%%%%%%%%%%%%%%%%%%%%%% Slide
\begin{frame}{Data processing}

	\textbf{Row indexing}

	\begin{itemize}
		\item Indexing the i-th and j-th non-continuous rows: \texttt{df.iloc[[i, j]]}
	\end{itemize}

	\begin{figure}
		\centering
		\includegraphics[width=0.5\linewidth]{img/non-consecutive_rows.png}
	\end{figure}

	Note that the index of the resulting dataframe is not necessarily sorted, it keeps the order in which we selected the rows to subset

\end{frame}

%%%%%%%%%%%%%%%%%%%%%%%%%%%%%%%%%%%%%%%%%%% Slide
\begin{frame}{Data processing}

	\textbf{Row indexing}

	The second method to index rows in Pandas is \texttt{.loc[]}. It subsets the rows whose index values coincide with the input inside the brackets.

	\begin{itemize}
		\item Until now, the dataframes we've worked with had and index which coincided with the row number
		\item That's not always the case, as we'll soon see
		\item The command to index the row whose index value is \texttt{i} is this: \texttt{df.loc[i]}
	\end{itemize}

We'll show more on the diference between the \texttt{.loc[]} and \texttt{iloc[]} methods in the next slide

\end{frame}

%%%%%%%%%%%%%%%%%%%%%%%%%%%%%%%%%%%%%%%%%%% Slide
\begin{frame}{Data processing}

	\textbf{Row indexing}

	\begin{figure}
		\centering
		\includegraphics[width=\linewidth]{img/row_loc_iloc.png}
	\end{figure}

\end{frame}

%%%%%%%%%%%%%%%%%%%%%%%%%%%%%%%%%%%%%%%%%%% Slide
\begin{frame}{Data processing}

	\textbf{Single-value indexing}

	To index a single value of a dataframe, we need to index a column and the row-index value \texttt{.loc[]} or position \texttt{.iloc[]}

	\begin{multicols}{2}

		\begin{figure}
			\centering
			\includegraphics[width=\linewidth]{img/single_value_index1.png}
		\end{figure}
		\begin{figure}
			\centering
			\includegraphics[width=\linewidth]{img/single_value_index2.png}
		\end{figure}

	\end{multicols}

	We can specify the row first and the column later, or viceversa

\end{frame}

%%%%%%%%%%%%%%%%%%%%%%%%%%%%%%%%%%%%%%%%%%% Subsection
\subsection{Filtering}

%%%%%%%%%%%%%%%%%%%%%%%%%%%%%%%%%%%%%%%%%%% Slide
\begin{frame}{Data processing}

	\textbf{Filtering}

	\begin{itemize}
		\item In Stata, we use the command \texttt{keep if \textit{column\_condition}} to filter observations
		\item Pandas' syntax to filter is heavier, as we shall see soon
	\end{itemize}

\end{frame}

%%%%%%%%%%%%%%%%%%%%%%%%%%%%%%%%%%%%%%%%%%% Slide
\begin{frame}{Data processing}

	\textbf{Filtering}

	\begin{itemize}
		\item To filter values of a dataframe, we use brackets and include a list or Pandas series with boolean values inside them:
	\end{itemize}

	\hspace{7mm} \texttt{df[list\_with\_booleans]}

	\begin{itemize}	
		\item The observations filtered-in are the ones that have a value of \texttt{True} in their cooresponding position
	\end{itemize}

\end{frame}

%%%%%%%%%%%%%%%%%%%%%%%%%%%%%%%%%%%%%%%%%%% Slide
\begin{frame}{Data processing}

	\textbf{Filtering}

	Note that the list or Pandas series with booleans needs to have the same length as the dataframe

	\begin{multicols}{2}

		\begin{figure}
			\centering
			\includegraphics[width=0.85\linewidth]{img/boolean_list.png}
		\end{figure}
		\begin{figure}
			\centering
			\includegraphics[width=\linewidth]{img/crops_filtered.png}
		\end{figure}

	\end{multicols}

\end{frame}

%%%%%%%%%%%%%%%%%%%%%%%%%%%%%%%%%%%%%%%%%%% Slide
\begin{frame}{Data processing}

	\textbf{Filtering}

	\begin{itemize}
		\item Other than a list with booleans, we can use a Pandas series with booleans
		\item The advantage of this is that we can generate them very easily when operating a dataframe column with a logical condition
	\end{itemize}

	\begin{figure}
		\centering
		\includegraphics[width=0.45\linewidth]{img/boolean_series.png}
	\end{figure}


\end{frame}

%%%%%%%%%%%%%%%%%%%%%%%%%%%%%%%%%%%%%%%%%%% Slide
\begin{frame}{Data processing}

	\textbf{Filtering}

	\begin{figure}
		\centering
		\includegraphics[width=0.6\linewidth]{img/fewest_crops.png}
	\end{figure}

\end{frame}

%%%%%%%%%%%%%%%%%%%%%%%%%%%%%%%%%%%%%%%%%%% Slide
\begin{frame}{Data processing}

	\textbf{Filtering}

	We can also use more than one condition at the same time:

	\begin{figure}
		\centering
		\includegraphics[width=0.9\linewidth]{img/crops_subset.png}
	\end{figure}

	\textbf{Important:} When using more than one condition, each of them must be enclosed in parentheses.

\end{frame}

%%%%%%%%%%%%%%%%%%%%%%%%%%%%%%%%%%%%%%%%%%% Slide
\begin{frame}{Data processing}

	\textbf{Creating new columns}

	\begin{itemize}
		\item To create a new column in a dataframe, we define it using the brackets as in:
	\end{itemize}
	
	\hspace{7mm} \texttt{df[new\_col\_name] = value}
	
	\begin{itemize}
		\item Other than a value, we can use columns operations to define new columns
	\end{itemize}

\end{frame}

%%%%%%%%%%%%%%%%%%%%%%%%%%%%%%%%%%%%%%%%%%% Slide
\begin{frame}{Data processing}

	\textbf{Creating new columns}

	\begin{figure}
		\centering
		\includegraphics[width=0.5\linewidth]{img/column_generation.png}
	\end{figure}

\end{frame}

%%%%%%%%%%%%%%%%%%%%%%%%%%%%%%%%%%%%%%%%%%% Slide
\begin{frame}{Data processing}

	\textbf{Group by}

	\begin{itemize}
		\item The syntax to group a dataframe is:
	\end{itemize}

	\hspace{7mm} \texttt{df.groupby(by = col\_name).sum()}

	\begin{itemize}
		\item This will return a grouped dataframe by \texttt{col\_name}, where every other column contains a sum of its previous values by \texttt{col\_name}
		\item Other possible operations are: \texttt{.mean()}, \texttt{.std()}, and \texttt{.quantile()}
		\item We can also group by more than one column, by replacing \texttt{col\_name} with a list of strings containing the column names to group by
	\end{itemize}

\end{frame}

%%%%%%%%%%%%%%%%%%%%%%%%%%%%%%%%%%%%%%%%%%% Slide
\begin{frame}{Data processing}

	\textbf{Group by}

	\begin{figure}
		\centering
		\includegraphics[width=\linewidth]{img/groupby.png}
	\end{figure}

\end{frame}

%%%%%%%%%%%%%%%%%%%%%%%%%%%%%%%%%%%%%%%%%%% Slide
\begin{frame}{Data processing}

	\textbf{Group by}

	\begin{multicols}{2}
		\begin{itemize}
			\item After grouping, the resulting dataframe has the group column as index
			\item This means that \texttt{farmer\_revenue} has a \textbf{meaningful index} now, an index that has information itself and is different than the row number
			\item Meaningful indices can be useful in some cases, but that's out of the topics we'll cover today
		\end{itemize}
		\begin{figure}
			\centering
			\includegraphics[width=0.8\linewidth]{img/farmer_revenue.png}
		\end{figure}
	\end{multicols}{2}

\end{frame}

%%%%%%%%%%%%%%%%%%%%%%%%%%%%%%%%%%%%%%%%%%% Slide
\begin{frame}{Data processing}

	\textbf{Group by}

	To move \texttt{farmerid} back to the columns, use the attribute \texttt{.reset\_index()}. We could have also used the argument \texttt{as\_index=False} in \texttt{.groupby()} in the first place.

	\begin{multicols}{2}
		\begin{figure}
			\centering
			\includegraphics[width=0.8\linewidth]{img/reset_index.png}
		\end{figure}
		\begin{figure}
			\centering
			\includegraphics[width=\linewidth]{img/groupby_no_index.png}
		\end{figure}
	\end{multicols}{2}

\end{frame}

%%%%%%%%%%%%%%%%%%%%%%%%%%%%%%%%%%%%%%%%%%% Slide
\begin{frame}{Data processing}

\end{frame}

%%%%%%%%%%%%%%%%%%%%%%%%%%%%%%%%%%%%%%%%%%% Slide
\begin{frame}{Data processing}

\end{frame}

\end{document}
